\documentclass[12pt, a4paper, oneside]{memoir}
\usepackage{hyperref}
\usepackage{graphicx}
\usepackage{epsfig}
\usepackage{amsmath}
\usepackage{amssymb}
\usepackage{amsthm}
\usepackage{booktabs}
\usepackage{stmaryrd}
\usepackage{url}
\usepackage{float}
\usepackage{longtable}
\usepackage[figuresright]{rotating}
\usepackage[utf8]{inputenc}
\usepackage[T1]{fontenc}
\usepackage{lmodern}
\usepackage[polish]{babel}
\usepackage{indentfirst}
\usepackage{eso-pic}
\usepackage[ruled,vlined]{algorithm2e}
\usepackage{algorithmicx}
\usepackage{listings}
\usepackage{listingsutf8}
\usepackage{citeref}
\usepackage{color}
\usepackage{memhfixc}
\usepackage{pdfpages}
\usepackage{copyrightbox}
\usepackage{attachfile2}
\usepackage{siunitx}
\usepackage{currency}

\renewcommand\familydefault{\sfdefault}
\renewcommand{\bibitempages}[1]{\newblock {\scriptsize [\mbox{str.\ }#1]}}
\renewcommand{\emph}[1]{\textit{#1}}
\renewcommand\sf{\sffamily}

\definecolor{greenyellow}   {cmyk}{0.15,    0, 0.69,    0}
\definecolor{yellow}        {cmyk}{   0,    0,    1,    0}
\definecolor{goldenrod}     {cmyk}{   0, 0.10, 0.84,    0}
\definecolor{dandelion}     {cmyk}{   0, 0.29, 0.84,    0}
\definecolor{apricot}       {cmyk}{   0, 0.32, 0.52,    0}
\definecolor{peach}         {cmyk}{   0, 0.50, 0.70,    0}
\definecolor{melon}         {cmyk}{   0, 0.46, 0.50,    0}
\definecolor{yelloworange}  {cmyk}{   0, 0.42,    1,    0}
\definecolor{orange}        {cmyk}{   0, 0.61, 0.87,    0}
\definecolor{burntorange}   {cmyk}{   0, 0.51,    1,    0}
\definecolor{bittersweet}   {cmyk}{   0, 0.75,    1, 0.24}
\definecolor{redorange}     {cmyk}{   0, 0.77, 0.87,    0}
\definecolor{mahogany}      {cmyk}{   0, 0.85, 0.87, 0.35}
\definecolor{maroon}        {cmyk}{   0, 0.87, 0.68, 0.32}
\definecolor{brickred}      {cmyk}{   0, 0.89, 0.94, 0.28}
\definecolor{red}           {cmyk}{   0,    1,    1,    0}
\definecolor{orangered}     {cmyk}{   0,    1, 0.50,    0}
\definecolor{rubinered}     {cmyk}{   0,    1, 0.13,    0}
\definecolor{wildstrawberry}{cmyk}{   0, 0.96, 0.39,    0}
\definecolor{salmon}        {cmyk}{   0, 0.53, 0.38,    0}
\definecolor{carnationpink} {cmyk}{   0, 0.63,    0,    0}
\definecolor{magenta}       {cmyk}{   0,    1,    0,    0}
\definecolor{violetred}     {cmyk}{   0, 0.81,    0,    0}
\definecolor{rhodamine}     {cmyk}{   0, 0.82,    0,    0}
\definecolor{mulberry}      {cmyk}{0.34, 0.90,    0, 0.02}
\definecolor{redviolet}     {cmyk}{0.07, 0.90,    0, 0.34}
\definecolor{fuchsia}       {cmyk}{0.47, 0.91,    0, 0.08}
\definecolor{lavender}      {cmyk}{   0, 0.48,    0,    0}
\definecolor{thistle}       {cmyk}{0.12, 0.59,    0,    0}
\definecolor{orchid}        {cmyk}{0.32, 0.64,    0,    0}
\definecolor{darkorchid}    {cmyk}{0.40, 0.80, 0.20,    0}
\definecolor{purple}        {cmyk}{0.45, 0.86,    0,    0}
\definecolor{plum}          {cmyk}{0.50,    1,    0,    0}
\definecolor{violet}        {cmyk}{0.79, 0.88,    0,    0}
\definecolor{royalpurple}   {cmyk}{0.75, 0.90,    0,    0}
\definecolor{blueviolet}    {cmyk}{0.86, 0.91,    0, 0.04}
\definecolor{periwinkle}    {cmyk}{0.57, 0.55,    0,    0}
\definecolor{cadetblue}     {cmyk}{0.62, 0.57, 0.23,    0}
\definecolor{cornflowerblue}{cmyk}{0.65, 0.13,    0,    0}
\definecolor{midnightblue}  {cmyk}{0.98, 0.13,    0, 0.43}
\definecolor{navyblue}      {cmyk}{0.94, 0.54,    0,    0}
\definecolor{royalblue}     {cmyk}{   1, 0.50,    0,    0}
\definecolor{blue}          {cmyk}{   1,    1,    0,    0}
\definecolor{cerulean}      {cmyk}{0.94, 0.11,    0,    0}
\definecolor{cyan}          {cmyk}{   1,    0,    0,    0}
\definecolor{processblue}   {cmyk}{0.96,    0,    0,    0}
\definecolor{skyblue}       {cmyk}{0.62,    0, 0.12,    0}
\definecolor{turquoise}     {cmyk}{0.85,    0, 0.20,    0}
\definecolor{tealblue}      {cmyk}{0.86,    0, 0.34, 0.02}
\definecolor{aquamarine}    {cmyk}{0.82,    0, 0.30,    0}
\definecolor{bluegreen}     {cmyk}{0.85,    0, 0.33,    0}
\definecolor{emerald}       {cmyk}{   1,    0, 0.50,    0}
\definecolor{junglegreen}   {cmyk}{0.99,    0, 0.52,    0}
\definecolor{seagreen}      {cmyk}{0.69,    0, 0.50,    0}
\definecolor{green}         {cmyk}{   1,    0,    1,    0}
\definecolor{forestgreen}   {cmyk}{0.91,    0, 0.88, 0.12}
\definecolor{pinegreen}     {cmyk}{0.92,    0, 0.59, 0.25}
\definecolor{limegreen}     {cmyk}{0.50,    0,    1,    0}
\definecolor{yellowgreen}   {cmyk}{0.44,    0, 0.74,    0}
\definecolor{springgreen}   {cmyk}{0.26,    0, 0.76,    0}
\definecolor{olivegreen}    {cmyk}{0.64,    0, 0.95, 0.40}
\definecolor{rawsienna}     {cmyk}{   0, 0.72,    1, 0.45}
\definecolor{sepia}         {cmyk}{   0, 0.83,    1, 0.70}
\definecolor{brown}         {cmyk}{   0, 0.81,    1, 0.60}
\definecolor{tan}           {cmyk}{0.14, 0.42, 0.56,    0}
\definecolor{gray}          {cmyk}{   0,    0,    0, 0.50}
\definecolor{black}         {cmyk}{   0,    0,    0,    1}
\definecolor{white}         {cmyk}{   0,    0,    0,    0}
\definecolor{mygray}        {cmyk}{0.03, 0.03, 0.03, 0.03}

\bibliographystyle{plain}

\lstset{
	basicstyle=\footnotesize\ttfamily,
	keywordstyle=\color{blue},
	stringstyle=\color{violetred},
	commentstyle=\color{green}, 
	frame=single,
	frameround=tttt,
	framesep=5pt,
    inputencoding=utf8/cp1250,
    breaklines=true,
    postbreak=\mbox{\textcolor{red}{$\hookrightarrow$}\space},
}

\ifpdf
    \hypersetup{
        plainpages=false,
        bookmarksnumbered,
        colorlinks=true,
        linkcolor=black,
        citecolor=black,
        filecolor=black,
        urlcolor=black,
        unicode
    }
    \pdfimageresolution=600
    \pdfcompresslevel=9
    \usepackage{thumbpdf}
\fi

\attachfilesetup{
    color=black
}

\newcounter{ax}
\newlistof{listofappendices}{loa}{Spis załączników}
\DeclareDocumentCommand{\ax}{ m m o }{%
    \refstepcounter{ax}%
    \textattachfile[#3]{#2}{#1~[F\theax]}%
    \addcontentsline{loa}{ax}{\protect\numberline{F\theax}#1}%
}
\makeatletter
\let\l@ax\l@figure
\makeatother

\sisetup{locale=PL}
\DefineCurrency{PLN}{
    iso=PLN,
    kind=symbol,
    name={złoty},
    plural={złotych},
    symbol={zł},
}

\settypeblocksize{23cm}{16cm}{*}

\setlrmargins{*}{2 cm}{*}
\setulmargins{*}{*}{1.3}

\setheadfoot{\onelineskip}{2\onelineskip}
\setheaderspaces{*}{2\onelineskip}{*}

\renewcommand\baselinestretch{1.3}

\checkandfixthelayout

\makechapterstyle{mychapterstyle}{%
    \renewcommand{\chapnamefont}{\LARGE\sffamily\bfseries}%
    \renewcommand{\chapnumfont}{\LARGE\sffamily\bfseries}%
    \renewcommand{\chaptitlefont}{\Huge\sffamily\bfseries}%
    \renewcommand{\printchaptertitle}[1]{%
        \chaptitlefont\hrule height 0.5 pt \vspace{1em}%
        {##1}\vspace{1em}\hrule height 0.5 pt%
    }%
    \renewcommand{\printchapternum}{%
        \chapnumfont\thechapter%
    }%
}

\chapterstyle{mychapterstyle}

\setsecheadstyle{\Large\sffamily\bfseries}
\setsubsecheadstyle{\large\sffamily\bfseries}
\setsubsubsecheadstyle{\normalfont\sffamily\bfseries}
\setparaheadstyle{\normalfont\sffamily}

\makeevenhead{headings}{\thepage}{}{\small\slshape\leftmark}
\makeoddhead{headings}{\small\slshape\rightmark}{}{\thepage}

\settocdepth{subsection}

\setsecnumdepth{subsection}
\maxsecnumdepth{subsection}
\settocdepth{subsection}
\maxtocdepth{subsection}

\setlength{\epigraphwidth}{0.57\textwidth}
\setlength{\epigraphrule}{0pt}
\setlength{\beforeepigraphskip}{1\baselineskip}
\setlength{\afterepigraphskip}{2\baselineskip}

\setlength{\parskip}{2mm}

\newcommand{\epitext}{\sffamily\itshape}
\newcommand{\epiauthor}{\sffamily\scshape ---~}
\newcommand{\epititle}{\sffamily\itshape}
\newcommand{\epidate}{\sffamily\scshape}
\newcommand{\episkip}{\medskip}

\newcommand{\myepigraph}[4]{%
	\epigraph{\epitext #1\episkip}{\epiauthor #2\\\epititle #3 \epidate(#4)}\noindent%
}

\renewcommand{\thefootnote}{\fnsymbol{footnote}}

\renewcommand{\today}{
    \ifcase \month%
        \or Styczeń\or Luty\or Marzec\or Kwiecień\or Maj \or Czerwiec\or Lipiec\or Sierpień\or Wrzesień\or Październik\or Listopad\or Grudzień%
    \fi~ \number \year%
}

\let\mainmatterorig\mainmatter
\renewcommand\mainmatter{%
    \edef\temppagenumber{\arabic{page}}%
	\mainmatterorig
	\setcounter{page}{\temppagenumber}%
}

\newcommand{\link}[1]{\href{https://#1}{\nolinkurl{#1}}}
\newcommand{\RNum}[1]{\uppercase\expandafter{\romannumeral #1\relax}}

\begin{document}
\frontmatter

% *************** Strona tytułowa ***************
\pagestyle{empty}
\pagenumbering{arabic}

\noindent

\AddToShipoutPictureBG*{
	\AtPageUpperLeft{\raisebox{-\height}{
		\includegraphics[height=0.23\paperwidth]{media/pbs_logo}
	}
}}

\vspace*{4cm}

\begin{tabular}{l}
   \HUGE PRACA DYPLOMOWA \\
   \HUGE INŻYNIERSKA \\
   \LARGE na~kierunku Informatyka Stosowana
\end{tabular}

\vfill

\begin{tabular}{p{15cm}}
    \Large\textbf{Opracowanie zestawu ćwiczeń laboratoryjnych wykorzystujących Raspberry~Pi} \\
	\\
	\large\textbf{Development of~a~set of~laboratory exercises using Raspberry~Pi}
\end{tabular}

\vfill

\begin{tabular}{l}
    \Large Remigiusz Dończyk \\
    \num{121179} \\
    \\
    \\
    dr.~inż.~Beata Marciniak \\
    \\
    Bydgoszcz, \today
\end{tabular}

\cleardoublepage

\pagestyle{plain}
\scriptsize

\section*{Metryka pracy dyplomowej}
\subsubsection*{Dane ogólne}
\begin{tabular}{p{3.5cm}p{11.5cm}}
    Nazwa Uczelni   & Politechnika Bydgoska im.~Jana i~Jędrzeja Śniadeckich \\
    Wydział         & Telekomunikacji, Informatyki i~Elektrotechniki \\
    Kierunek        & Informatyka Stosowana \\
    Tryb studiów    & niestacjonarne \\
    Dane autora     & Remigiusz Dończyk, \num{121179} \\
    Dane promotora  & dr.~inż.~Beata Marciniak
\end{tabular}

\subsubsection*{Dane dotyczące pracy dyplomowej}
\begin{tabular}{p{3.5cm}p{11.5cm}}
    Język pracy     & język polski [PL] \\
    Tytuł pracy     & Opracowanie zestawu ćwiczeń laboratoryjnych wykorzystujących \emph{Raspberry Pi}. \\
    Opis pracy      &%
        Celem pracy jest dobór zestawów laboratoryjnych i~opracowanie instrukcji do~maksymalnie \num{6}~ćwiczeń
        z~użyciem dostarczonych przez prowadzącego zestawów czujników oraz \emph{Raspberry~Pi}.
        Każde gotowe ćwiczenie powinno składać~się z~przykładowego programu w~języku \emph{Python}, pozwalającego
        na~konfigurację urządzeń, oraz~instrukcji przebiegu ćwiczenia.
        \begin{enumerate}
            \item Dobór zestawów czujników, różnych dla~każdego ćwiczenia.
            \item Przygotowanie programów w~języku \emph{Python} wykorzystujących zestawy czujników.
            \item Opracowanie instrukcji ćwiczeń do~każdego zestawu, zawierających przykładowe zadanie do~rozwiązania.
            \item Wykonanie rozwiązań zadań zdefiniowanych w~instrukcjach ćwiczeń.
            \item Przetestowanie działania rozwiązań i~bezpieczeństwa użytkowania.
        \end{enumerate} \\
    Typ pracy       & inżynierska \\
    Streszczenie    &%
        Praca dyplomowa jest rozwiązaniem problemu inżynierskiego na~temat wykorzystania mikrokomputera
        \emph{Raspberry Pi} wraz~z~zestawem modułów fizycznych do~opracowania serii zadań laboratoryjnych.
        Na~początku pracy zamieszczono informacje wspólne dla~laboratoriów: wymieniono dostępne moduły i~wybór
        spośród~nich zestawów laboratoryjnych; w~przeglądzie rynku wyróżniono istniejące alternatywy
        \emph{Raspberry~Pi}; opisano połączenia fizyczne i~logiczne wykorzystywane przez~użyte~moduły, a~także
        oprogramowanie składające~się na~wygodny w~użytkowaniu dla~prowadzącego i~studentów system, oraz~rozwiązania
        mające~wpływ na~bezpieczeństwo urządzeń.
        Dalsza część pracy jest~podzielona na~sekcje poświęcone każdemu z~opracowanych laboratoriów, prezentuje zestaw
        laboratoryjny wraz~z~rozwiązaniem zadania laboratoryjnego w~celu weryfikacji poprawności przygotowania. \\
    Słowa kluczowe  & komputer, \emph{Python}, \emph{Raspberry Pi}, czujniki, laboratoria
\end{tabular}

\newpage

\section*{Diploma thesis record}
\subsubsection*{General information}
\begin{tabular}{p{3.5cm}p{11.5cm}}
    University name           & Bydgoszcz University of~Science and~Technology \\
    Faculty                   & Telecommunications, Computer~Science and~Electrical~Engineering \\
    Field of study            & Applied Computer Science \\
    Mode of study             & part-time \\
    Author's information      & Remigiusz Dończyk, \num{121179} \\
    Supervisor's information  & dr.~inż.~Beata Marciniak
\end{tabular}

\subsubsection*{Data regarding the diploma thesis}
\begin{tabular}{p{3.5cm}p{11.5cm}}
    Language of thesis  & Polish [PL] \\
    Title of thesis     & Development of~a~set of~laboratory exercises using \emph{Raspberry Pi}. \\
    Description         &%
        The goal of~the~thesis is~the~selection of~laboratory exercise kits and~the~development of~up~to~\num{6}
        experiments using hardware sensors and~\emph{Raspberry Pi} boards provided by~the~instructor.
        Each laboratory should consist of~an~example \emph{Python} program, allowing for~the~configuration of~hardware,
        as~well as~an~exercise guide document.
        \begin{enumerate}
            \item Selection of~sensors, unique to~each exercise.
            \item Preparation of~\emph{Python} programs utilizing the~selected hardware.
            \item Development of~instructions for~each exercise, containing tasks to~be~completed.
            \item Execution of~solutions for~tasks contained in~the~instructions.
            \item Test of~the~solutions' functionality and~safety of~usage.
        \end{enumerate} \\
    Type of thesis      & Engineer’s \\
    Abstract            &%
      This~thesis is~the~solution to~an~engineering problem involving the~use~of a~\emph{Raspberry Pi} microcomputer
      and~a~set of~physical modules to~develop a~series of~laboratory exercises.
      At~the~beginning of~the~document is~information common to~the~laboratories: a~list of~the~available modules
      and~the~selection of~laboratory kits from~them; the~market survey, an~overview of~existing alternatives
      to~\emph{Raspberry Pi}; a~description of~the~physical and~logical connections used by~the~selected modules,
      as~well~as the~software making~up a~user-friendly system for~the~instructor and~students, along with~solutions
      affecting device security.
      The~latter part of~the~thesis is~divided into~sections dedicated to~each~laboratory, presents the~laboratory kit
      and~the~solution to~the~laboratory exercise to~verify its~correctness. \\
    Keywords            & computer, \emph{Python}, \emph{Raspberry Pi}, sensors, laboratories
\end{tabular}

\normalsize\clearpage
\tableofcontents

\mainmatter
\addtocounter{page}{1}
% CONTENT START

\chapter{Informacje wspólne}\label{ch:info}

\section{Dostępny sprzęt}\label{sec:harware}

\subsection{Raspberry Pi}\label{subsec:rpi}

Opis

Raspberry Pi kontra Arduino

\subsection{Moduły}\label{subsec:modules}

Dostępne prowadzącemu moduły:
\begin{itemize}
  \item Sense HAT~\cite{sense}
  \item Flick HAT~\cite{flick}
  \item Grove Base Kit~\cite{grove}
  \item Automation HAT Mini~\cite{automation}
  \item Enviro HAT~\cite{enviro}
  \item Unicorn HAT~\cite{unicorn}
  \item Traffic pHAT~\cite{traffic}
\end{itemize}

Niektóre moduły zostały wyłączone z~procesu selekcji:
\begin{itemize}
  \item Flick HAT -- firma producenta została rozwiązana~\cite{pisupply}, oficjalna dokumentacja ich~produktów
        nie~jest dostępna, a~konfiguracja nakładki nie~jest oczywista.
  \item Automation HAT Mini -- jest zaprojektowany do~pracy z~urządzeniami o~napięciu 12V lub 24V,
        co~przy nieprawidłowym użytkowaniu może uszkodzić płytkę.
  \item Unicorn HAT -- jest wyposażony w~macierz diod 8~x~8, podobnie do~Sense HAT, który~ma~szersze możliwości.
\end{itemize}

Przydział pozostałych modułów opisany jest w~sekcjach przeznaczonych dla~każdego z~laboratoriów.
W związku z~ilością możliwych zastosowań zestawu Grove,
został on~wykorzystany (używając różne urządzenia zewnętrzne) dwukrotnie.

\section{Połączenia fizyczne}\label{sec:connector}

\subsection{GPIO}\label{subsec:gpio}

Lorem ipsum

\subsection{Grove}\label{subsec:grove}

Moduł \emph{Grove Base HAT} jest~podłączony do~płytki \emph{Raspberry~Pi} przy pomocy interfejsu \emph{GPIO}
i~pełni funkcję adaptera tego~interfejsu na~własny system złączy~\cite{grovedesc}.
Moduł wyprowadza interfejs \emph{GPIO}, jednak wiele połączeń pokrywa się z~systemem \emph{Grove},
w~związku z~czym wykorzystanie ich nie~jest bezpieczne bez~wcześniejszego zrozumienia tego~systemu.
Na~module \emph{Grove Base HAT} złącza podpisane są~przy użyciu numeracji \emph{BCM}.

Połączenia fizyczne wykorzystywane przez interfejs \emph{Grove} są~podobne do~standardowych połączeń \emph{I²C}\@.
Połączenia \emph{VCC} i~\emph{GND} zawsze dostarczają elementom zasilanie,
jednak pozostałe~dwa zachowują~się w~różny sposób w~zależności od~rodzaju złącza.
Połączenia te~wykorzystują interfejs \emph{GPIO} do~komunikacji z~płytką, więc~wykorzystanie danego portu \emph{Grove}
wraz z~pinami \emph{GPIO} o~tej~samej numeracji wiąże~się z~konfliktem.

\begin{figure}[ht]
  \caption{Diagram złączy \emph{Grove Base HAT}}
  \label{fig:grovehat}
  \centering
  \copyrightbox{
    \includegraphics[height=7cm]{media/botland_GroveBaseHat}
  }{
    \textcopyright Botland B. Derkacz Sp. k.
  }
\end{figure}

\begin{table}[ht]
  \caption{Opis złączy \emph{Grove Base HAT}}
  \label{tab:grove}
  \begin{tabularx}{\linewidth}{@{}lX@{}}
    GPIO      & Bezpośrednie połączenie do złącza \emph{GPIO} płytki \emph{Raspberry~Pi}.
                Może powodować konflikty ze złączami \emph{Grove}. \\
    I²C       & Zgodnie ze~standardowymi łączeniami, zewnętrzny pin pełni funkcję zegara, a~wewnętrzny przesyłu
                danych. \\
    Cyfrowy   & Zewnętrzny pin pełni funkcję pierwszego cyfrowego łącza wejścia/wyjścia, a~wewnętrzny jest łączem
                dodatkowym dla~modułów wykorzystujących dwie linie. \\
    UART      & Specjalizowana forma trybu cyfrowego, pierwszy pin odbiera dane od~podłączonego urządzenia, a~drugi
                wysyła. \\
    Analogowy & Wejścia analogowe są~podłączone do~wbudowanego przetwornika analogowo-cyfrowego, co~pozwala
                na~ich~użycie z~płytką \emph{Raspberry~Pi}, nie~posiadającej natywnie takiej możliwości.
                Podobnie do~trybu cyfrowego, linia zewnętrzna jest wykorzystywana w~pierwszej kolejności, a~wewnętrzna
                w~drugiej. \\
    PWM       & Port \emph{PWM} podłączony~jest do~sprzętowego złącza \emph{PWM} \emph{Raspberry~Pi}.
                Pozostałe złącza mogą~być używane jako~\emph{PWM} na~poziomie oprogramowania.
                Złącze sprzętowe pozwala na~lepszą dokładność przy~wysokich częstotliwościach, jednak wchodzi w~konflikt
                z~niektórymi innymi funkcjami płytki. \\
    SWD       & Dostęp do wbudowanego oprogramowania modułu Grove Base HAT\@.
  \end{tabularx}
\end{table}

\section{Połączenia logiczne}\label{sec:protocol}

\subsection{I²C}\label{subsec:i2c}

\subsection{I²S}\label{subsec:i2s}

\subsection{SPI}\label{subsec:spi}

\section{Oprogramowanie}\label{sec:software}

\subsection{Raspberry Pi OS}\label{subsec:rpi-os}

Dystrybucja GNU/Linux bazująca na dystrybucji Debian.

\subsection{Bash}\label{subsec:bash}

Powłoka wykorzystywana często w terminalach GNU/Linux wraz z językiem skryptowym.

\subsection{Python}\label{subsec:py}

Język skryptowy.

\subsection{rpi-lgpio}\label{subsec:lgpio}

Biblioteka do użytkowania interfejsu GPIO zastępująca RPi.GPIO\@.

\subsection{marimo}\label{subsec:marimo}

Środowisko-notatnik Python zastępujące Jupyter.

\section{Wdrożenie rozwiązania}\label{sec:deployment}

\subsection{Udostępnianie}\label{subsec:share}

Skrypt bash dostępny na moim serwerze, docelowo uczelnianym.

\subsection{Testy rozwiązań}\label{subsec:tests}

Testy rozwiązań studentów.

\subsection{Bezpieczeństwo}\label{subsec:security}

Opis bezpieczeństwa udostępniania i testów.

\chapter{Laboratorium I}\label{ch:lab-1}

\section{Wybrany sprzęt}\label{sec:lab1-hw}

\section{Dodatkowe oprogramowanie}\label{sec:lab1-sw}

\section{Rozwiązanie}\label{sec:lab1-sol}

\chapter{Laboratorium II}\label{ch:lab2}

\section{Wybrany sprzęt}\label{sec:lab2-hw}

\section{Dodatkowe oprogramowanie}\label{sec:lab2-sw}

\section{Rozwiązanie}\label{sec:lab2-sol}

\chapter{Laboratorium III}\label{ch:lab3}

\section{Wybrany sprzęt}\label{sec:lab3-hw}

\section{Dodatkowe oprogramowanie}\label{sec:lab3-sw}

\section{Rozwiązanie}\label{sec:lab3-sol}

\chapter{Laboratorium IV}\label{ch:lab4}

\section{Wybrany sprzęt}\label{sec:lab4-hw}

\section{Dodatkowe oprogramowanie}\label{sec:lab4-sw}

\section{Rozwiązanie}\label{sec:lab4-sol}

\chapter{Laboratorium V}\label{ch:lab5}

\section{Wybrany sprzęt}\label{sec:lab5-hw}

\section{Dodatkowe oprogramowanie}\label{sec:lab5-sw}

\section{Rozwiązanie}\label{sec:lab5-sol}

% CONTENT END
\bibliography{bibliography}
\backmatter\normalfont\clearpage
\listoffigures\clearpage
\listoftables

\appendix
\chapter{Załączniki}\label{ch:appendix}

TODO: Kod instalatora\ldots

\end{document}
