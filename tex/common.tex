\chapter{Informacje wstępne}\label{ch:info}

Rozdział ten skupia~się na~informacjach mających znaczenie dla~więcej niż~jednego z~laboratoriów.
W~związku z~tym, w~celu niepowtarzania tych informacji w~każdym istotnym miejscu, umieszczone~są~one w~przodującej
sekcji pracy.

\section{Przegląd rynku}\label{sec:market}

Na~rynku istnieją zestawy mogące stanowić poszczególne laboratoria, które mogłyby być~wykorzystane do~przygotowania
laboratoriów na~potrzeby uczelni.
Przystosowanie jednak niezależnych produktów wiąże~się z~komplikacjami wynikającymi z~różnic w~wykorzystywanych
rozwiązaniach technicznych oraz~programowych.

Ta~praca jest kolekcją kilku zestawów wykorzystujących sprzęt dostępny~już uczelni, a~także zrealizowanych w~jednolity
i~przystępny sposób.
Wygodna metoda instalacji, przygotowania, a~także wyczyszczenia środowiska laboratoryjnego oraz~wykonania zadań zgodna
dla~wszystkich realizowanych zagadnień niweluje te~komplikacje i~pozwala na~poświęcenie większej ilości czasu treści
zadań niźli zapoznania z~kolejnymi metodami interakcji ze~sprzętem.
Wykorzystanie opisanych poniżej rozwiązań pozwala również m.in.~na~automatyczne sprawdzanie poprawności rozwiązań.

\section{Dostępny sprzęt}\label{sec:harware}

\subsection{Raspberry Pi}\label{subsec:rpi}

\emph{Raspberry Pi}\thinspace\cite{pi} to~seria małych komputerów jednopłytkowych stworzonych przez Fundację
Raspberry~Pi w~Wielkiej Brytanii.
Celem Fundacji jest rozpowszechnienie systemów o~możliwościach obliczeniowych, co~wpływa na~niewielki koszt płytki,
i~prowadzi~do wykorzystania standardowej architektury procesora i~modułów komunikacyjnych (\emph{WiFi},
\emph{Bluetooth}).
Takie urządzenia można zastosować w wielu różnych projektach m.in.~do połączenia urządzeń Internetu Rzeczy.

Koszt najtańszego modelu wynosi ok.~\dPLN{20}, a~najtańszy pełnoprawny komputer w~serii kosztuje ok.~\dPLN{50}.

Urządzenia \emph{Raspberry Pi} wyposażone~są w~zestaw czterdziestu pinów \emph{GPIO} pozwalających na~podłączenie
dowolnych układów wykorzystujących zasilanie \qty{3.3}{\V} lub \qty{5}{\V}\@.
Stanowią dzięki~temu przystępną opcję zarówno do~nauki, użytkowania systemów komputerowych, jak~i~eksperymentów
polegających na budowaniu własnych urządzeń.

\subsection{Moduły}\label{subsec:modules}

Moduły udostępnione przez promotora na~potrzeby niniejszej pracy dyplomowej:
\begin{itemize}
  \item Sense HAT\thinspace\cite{sense};
  \item Flick HAT\thinspace\cite{flick};
  \item Grove Base Kit\thinspace\cite{grove};
  \item Automation HAT Mini\thinspace\cite{automation};
  \item Enviro HAT\thinspace\cite{enviro};
  \item Unicorn HAT\thinspace\cite{unicorn};
  \item Traffic pHAT\thinspace\cite{traffic}.
\end{itemize}

Niektóre moduły nie~zostały uwzględnione w~procesie selekcji -- tab.~\ref{tab:excluded}.

\begin{table}[H]
  \caption{Spis niewykorzystanych modułów}
  \label{tab:excluded}
  \begin{tabularx}{\linewidth}{@{}lX@{}}
    \textbf{Moduł}       & \textbf{Powód} \\
    \midrule
    Flick HAT            & Firma producenta została rozwiązana\thinspace\cite{pisupply}, oficjalna dokumentacja
                           ich~produktów nie~jest dostępna, a~konfiguracja nakładki nie~jest oczywista. \\
    \midrule
    Automation HAT Mini  & Zaprojektowany do~pracy z~urządzeniami o~napięciu \qty{12}{\V} lub \qty{24}{\V},
                           co~przy nieprawidłowym użytkowaniu może uszkodzić płytkę. \\
    \midrule
    Unicorn HAT          & Wyposażony w~macierz diod $8\,\times\,8$, podobnie do~\emph{Sense HAT}, który~ma~szersze
                           możliwości.
  \end{tabularx}
\end{table}

Pozostałe moduły zostały rozdzielone do wykorzystania w~poszczególnych laboratoriach.
W związku z~ilością możliwych zastosowań zestawu \emph{Grove} został on~wykorzystany (używając różnych~urządzeń
zewnętrznych) dwukrotnie.
Szczegółowe opisy połączeń i~wykorzystania znajdują~się w~rozdziałach poświęconych każdemu z~laboratoriów z~osobna.

\section{Uzupełnienie: Alternatywy Raspberry Pi}\label{sec:alternate}

Wspominając o~tym urządzeniu nie~można zapomnieć o~podobnych urządzeniach dostępnych na~rynku.
W~ostatnich latach na~rynku pojawiły się~bowiem alternatywy o~porównywalnej lub~nawet korzystniejszej cenie.
Niektórymi z~popularnych rozwiązań w~tej~kategorii są~\emph{Banana Pi}\thinspace\cite{bananapi},
\emph{Odroid}\thinspace\cite{odroid}, \emph{Libre Computer}\thinspace\cite{libre},
\emph{Orange Pi}\thinspace\cite{orangepi}, lub~poleasingowe komputery w~małej obudowie.

\emph{Raspberry Pi} pełniło ważną funkcję w~popularyzowaniu pojęcia komputerów jednopłytkowych jako~niedrogiego
narzędzia do~nauki i~prototypowania, i nadal jest najlepiej udokumentowanym z~konkurentów, co~jest ważnym elementem
przystępności w~środowisku naukowym.
Alternatywy często skupiają~się w~pierwszej kolejności na~udostępnieniu wyższej funkcjonalności, co~sprawia,~że
są~dobrą opcją dla~bardziej zaawansowanych użytkowników.

Osobnym rodzajem popularnych urządzeń jest~seria \emph{Arduino}, która jest~często przytaczana
wraz~z~\emph{Raspberry Pi}.
Istnieją między nimi jednak istotne różnice, z~powodu których urządzenia~te nie~odgrywają tej~samej roli.
Urządzenia \emph{Arduino} nie~są pełnoprawnymi komputerami (wymagają osobnego urządzenia do zaprogramowania)
i~konkurują jedynie z~\emph{Raspberry Pi Pico}, a~nie~z~resztą płytek o~możliwości obsługi pełnego systemu operacyjnego.

\section{Połączenia fizyczne}\label{sec:connector}

\subsection{GPIO}\label{subsec:gpio}

Złącze \emph{General-Purpose Input/Output} jest stosowane w mikrokontrolerach, komputerach jednopłytkowych, i~innych
urządzeniach elektronicznych w~celu umożliwienia bezpośredniej komunikacji z~układami zewnętrznymi.
W~przypadku \emph{Raspberry Pi} składa~się z~czterdziestu pinów\thinspace\cite{pinout}.
Część możliwości wymaga konfiguracji jądra systemu, co~jest przez~producenta udostępnione w~przystępny sposób.
Rysunek~\ref{fig:pinout} zaznacza te~dodatkowe możliwości; piny wchodzące w~ich skład można również użyć
w~standardowy sposób.

\begin{figure}[H]
  \centering
  \copyrightbox{
    \includegraphics[width=0.9\linewidth]{media/pinoutxyz_bcmPinout}
  }{
    \textcopyright pinout.xyz; Philip~Howard, RogueM i~inni
  }
  \caption{Układ pinów \emph{Raspberry Pi} -- adresacja \emph{BCM}}
  \label{fig:pinout}
\end{figure}

Drugą dostępną metodą adresacji jest~adresacja fizyczna, rozpoczynająca~się od~\num{1}~i~\num{2} odnoszących~się
odpowiednio do~pinów \qty{3.3}{\V} i~\qty{5}{\V} po~lewej stronie powyższego rysunku, numerując kolejne kolumny
pinów.

Piny oznaczone mianem \emph{GCLK} udostępniają dostęp do~wbudowanego zegara sprzętowego.
\emph{PWM}, \emph{I²C}, \emph{UART}, i~\emph{SPI} zostaną omówione w~sekcji~\ref{sec:protocol} opisującej
połączenia logiczne.

Istnieje wiele bibliotek oferujących środowiska upraszczające interakcję z~urządzeniami wykorzystującymi interfejs
\emph{GPIO}, w~zależności od~języka programowania.
Praca~ta używa języka programowania \emph{Python}, więc przykłady wykorzystują wiązania \emph{liblgpio} dla~tego~języka
(opisane w~sekcji~\ref{subsec:lgpio}).

\subsection{Grove}\label{subsec:grove}

Moduł \emph{Grove Base HAT}\thinspace\cite{grovedesc} jest~podłączony do~płytki \emph{Raspberry~Pi} przy~pomocy
interfejsu \emph{GPIO} i~pełni funkcję adaptera tego~interfejsu na~własny system złączy.
Moduł wyprowadza interfejs \emph{GPIO}, jednak wiele połączeń pokrywa się z~systemem \emph{Grove},
w~związku z~czym wykorzystanie ich nie~jest bezpieczne bez~wcześniejszego zrozumienia tego~systemu.
Na~module \emph{Grove Base HAT} złącza podpisane są~przy użyciu numeracji \emph{BCM}.

Połączenia fizyczne wykorzystywane przez interfejs \emph{Grove} przedstawione na~rysunku~\ref{fig:grovehat} i~opisane
w~tabeli~\ref{tab:grove} są~podobne
do~standardowych połączeń \emph{I²C}\@.
Połączenia \emph{VCC} i~\emph{GND} zawsze dostarczają elementom zasilanie,
jednak pozostałe~dwa zachowują~się w~różny sposób w~zależności od~rodzaju złącza.
Połączenia te~wykorzystują interfejs \emph{GPIO} do~komunikacji z~płytką, więc~wykorzystanie danego portu \emph{Grove}
wraz z~pinami \emph{GPIO} o~tej~samej numeracji wiąże~się z~konfliktem.

\begin{figure}[H]
  \centering
  \copyrightbox{
    \includegraphics[width=0.6\linewidth]{media/botland_GroveBaseHat}
  }{
    \textcopyright Seeed Technology Co., Ltd.
  }
  \caption{Diagram złączy \emph{Grove Base HAT}}
  \label{fig:grovehat}
\end{figure}

\begin{table}[H]
  \caption{Opis złączy \emph{Grove Base HAT}}
  \label{tab:grove}
  \begin{tabularx}{\linewidth}{@{}lX@{}}
    \textbf{Złącze}  & \textbf{Opis} \\
    \midrule
    GPIO             & Bezpośrednie połączenie do złącza \emph{GPIO} płytki \emph{Raspberry~Pi}.
                       Może powodować konflikty ze złączami \emph{Grove}. \\
    \midrule
    I²C              & Zgodnie ze~standardowymi łączeniami, zewnętrzny pin pełni funkcję zegara, a~wewnętrzny przesyłu
                       danych. \\
    \midrule
    Cyfrowy          & Zewnętrzny pin pełni funkcję pierwszego cyfrowego łącza wejścia/wyjścia, a~wewnętrzny jest łączem
                       dodatkowym dla~modułów wykorzystujących dwie linie. \\
    \midrule
    UART             & Specjalizowana forma trybu cyfrowego, pierwszy pin odbiera dane od~podłączonego urządzenia,
                       a~drugi wysyła. \\
    \midrule
    Analogowy        & Wejścia analogowe są~podłączone do~wbudowanego przetwornika analogowo-cyfrowego, co~pozwala
                       na~ich~użycie z~płytką \emph{Raspberry~Pi}, nie~posiadającej natywnie takiej możliwości.
                       Podobnie do~trybu cyfrowego, linia zewnętrzna jest wykorzystywana w~pierwszej kolejności,
                       a~wewnętrzna w~drugiej. \\
    \midrule
    PWM              & Port \emph{PWM} podłączony~jest do~sprzętowego złącza \emph{PWM} \emph{Raspberry~Pi}.
                       Pozostałe złącza mogą~być używane jako~\emph{PWM} na~poziomie oprogramowania.
                       Złącze sprzętowe pozwala na~lepszą dokładność przy~wysokich częstotliwościach, jednak wchodzi
                       w~konflikt z~niektórymi innymi funkcjami płytki. \\
    \midrule
    SWD              & Dostęp do wbudowanego oprogramowania modułu \emph{Grove Base HAT}\@.
  \end{tabularx}
\end{table}

\section{Połączenia logiczne}\label{sec:protocol}

\subsection{PWM}\label{subsec:pwm}

\emph{Pulse Width Modulation}\thinspace\cite{pwm} to~technika polegająca na~kodowaniu wartości sygnału danych na~sygnale
nośnym prostokątnym o~stałej częstotliwości i~amplitudzie jako~czasu trwania poszczególnych impulsów elektrycznych,
przedstawiona na~rysunku~\ref{fig:pwm}.
Wartość przesyłaną ustanawia się~regulując proporcję czasu, w~której sygnał przyjmuje wartość wysoką.
Zbocze rosnące sygnału jest~wysyłane w~stałych odstępach czasowych, co~pozwala na~wykorzystanie sygnału
bez~transmisji osobnego sygnału synchronizacyjnego.
Dzięki zapisowi informacji poza amplitudą sygnału \emph{PWM} jest~odporne na~typowe szumy transmisyjne.

Wadą \emph{PWM} jest~zmienna zawartość energetyczna pulsów, w~związku z~czym linia transmisyjna i~urządzenia
korzystające z~tej~modulacji muszą~być odporne na~maksymalną możliwą ilość energii (wypełnienie sygnału bliskie
\qty{100}{\percent}), aby~zapewnić odporność na~uszkodzenia.
Cecha~ta znalazła jednak zastosowanie jako~prosta metoda kontroli ilości energii dostarczanej do~urządzeń o~zmiennych
zapotrzebowaniach, np.\ oświetlenia o~regulowanej jasności, bez~zmiany rzeczywistej wartości napięcia.

\begin{figure}[H]
  \centering
  \copyrightbox{
    \includegraphics[width=0.7\linewidth]{media/electronicspost_pwm}
  }{
    \textcopyright Electronics Post, Sasmita Barik
  }
  \caption{Modulacja szerokości impulsów}
  \label{fig:pwm}
\end{figure}

\subsection{I²C}\label{subsec:i2c}

\emph{Inter-Integrated Circuit}\thinspace\cite{i2c} to~synchroniczna szeregowa magistrala komunikacyjna pozwalająca
na~komunikację wielu urządzeń za~pomocą dwóch linii transmisyjnych.
Nazwa wywodzi~się z~wczesnego użycia jako~metoda komunikacyjna między układami scalonymi na~jednej płytce drukowanej.

Oryginalna idea I²C zakłada istnienie tylko jednego urządzenia nadającego, jednak dzięki podłączeniu urządzeń za~pomocą
otwartego drenu lub~kolektora oraz~istnieniu mechanizmu wykrywającego kolizje możliwa~jest praca w~systemie zawierającym
wiele urządzeń nadawczych.
Standardowa \num{7}-bitowa adresacja pozwala na~podłączenie \num{127}~urządzeń odbierających, jednak z~powodu
maksymalnej dopuszczalnej pojemności o~wartości \qty{400}{\pF} w~praktyce długość linii jest~ograniczona do~kilku
metrów, co~w~większości przypadków ogranicza ilość podłączonych urządzeń.
Dla sytuacji niestandardowych jest~również dostępna adresacja \num{10}-bitowa, zwiększająca limit
do~\num{1023}~odbiorców.

Obie linie są~w~stanie spoczynku utrzymywane w~stanie wysokim.
Rozpoczęcie transmisji jest sygnalizowane poprzez zmianę stanu przez nadajnik na~stan niski na~obu liniach,
a~następnie uruchomienie zegara i~przesłanie wiadomości o~formacie przedstawionym na~rysunku~\ref{fig:i2c}.
Wiadomość może zawierać jedną lub~więcej sekcję danych w~zależności od~rodzaju komunikacji.
Transmisja jest~zakończona poprzez powrót w~stan spoczynku.

\begin{figure}[H]
  \centering
  \copyrightbox{
    \includegraphics[width=0.9\linewidth]{media/soldered_i2c}
  }{
    \textcopyright Soldered Electronics d.o.o.
  }
  \caption{Przykład wiadomości na linii \emph{SDA} interfejsu \emph{I²C}}
  \label{fig:i2c}
\end{figure}

\subsection{I²S}\label{subsec:i2s}

\emph{Inter-Integrated Circuit Sound}\thinspace\cite{i2s} nie~jest powiązane z~\emph{I²C}, jednak ta~sama firma
utworzyła ich~specyfikacje.
Protokół~ten jest~skupiony na~wbudowanych urządzeniach audio, zapewniając prostą i~ograniczoną funkcjonalność w~celu
zapewnienia niezawodności i~prostoty wykorzystania.

Magistrala wykorzystuje trzy linie komunikacyjne, w~tym dwa zegary.
Pierwszy z~zegarów ustanawia rytm przesyłania bitów informacyjnych.
Drugi zegar jest~sygnałem wybierającym jeden z~dwóch dostępnych kanałów informacyjnych na~linii danych.
W~teorii sygnał~ten może~być niepodłączony do~zegara, jednak oba kanały przesyłają tą~samą ilość danych, w~związku
z~czym użycie odpowiednio dostosowanego zegara znacząco upraszcza implementację i~nie stosuje~się w~praktyce innych
rozwiązań.
Trzecia z~linii przesyła dane dźwiękowe szeregowo, zakodowane kodem uzupełnień do~dwóch, zaczynając od~najbardziej
znaczącego bitu danych jeden cykl po~przełączeniu kanału.
Rysunek~\ref{fig:i2s} prezentuje te zależności.

\begin{figure}[H]
  \centering
  \copyrightbox{
    \includegraphics[width=0.9\linewidth]{media/wikipedia_i2s}
  }{
    \textcopyright Walter Dvorak
  }
  \caption{Diagram czasowy \emph{I²S}}
  \label{fig:i2s}
\end{figure}

\subsection{SPI}\label{subsec:spi}

\emph{Serial Peripheral Interface}\thinspace\cite{spi} to~synchroniczny szeregowy interfejs komunikacyjny z~jednym
urządzeniem nadzorczym i~wieloma podrzędnymi wykorzystujący cztery linie.
Połączone urządzenia tworzą ze~swoich rejestrów wewnętrznych bufor kołowy, co~zapewnia prostą implementację
komunikacji w~trybie pełny dupleks.

Wykorzystywane linie pełnią funkcje opisane w~tabeli~\ref{tab:spilines}.

\begin{table}[H]
  \caption{Opis linii \emph{SPI}}
  \label{tab:spilines}
  \begin{tabularx}{\linewidth}{@{}lX@{}}
    \textbf{Linia}       & \textbf{Opis} \\
    \midrule
    Chip Select          & Sygnał aktywny w~stanie niskim przesyłany przez~urządzenie nadzorcze, uruchamiający
                           obsługę komend przez~podłączone urządzenia. \\
    \midrule
    Serial Clock         & Sygnał zegarowy nadawany przez~urządzenie nadzorcze. \\
    \midrule
    Master Out Slave In  & Linia wysyłająca dane urządzenia nadzorczego. \\
    \midrule
    Master In Slave Out  & Linia odbierająca dane urządzenia nadzorczego.
  \end{tabularx}
\end{table}

Przesłane komendy są~wykonywane w~momencie wykrycia zbocza rosnącego sygnału \emph{Chip Select}.
Standardowe~są dwa~sposoby połączeń wykorzystujące linie w~odmienny sposób.

W~pierwszej z~nich dla~$N$ urządzeń podrzędnych wymagane~jest~$N$ linii \emph{Chip Select}.
Na~potrzeby tej~pracy nadam tej~metodzie nazwę ``bezpośredniej'', gdyż~każde z~urządzeń podrzędnych jest~bezpośrednio
wybrane do~komunikacji przez~urządzenie nadzorcze, które przełącza w~tryb niski linię \emph{Chip Select} danego
urządzenia podrzędnego.
Linia \emph{MISO} jest~współdzielona, więc w~danym czasie wybrane może~być dokładnie jedno z~urządzeń.
Wszystkie urządzenia podrzędne otrzymują w~tym samym czasie te~same dane, gdyż~linia \emph{MOSI} również
jest~współdzielona, ale na~linii \emph{MISO} odpowiada jedynie wybrane urządzenie.
Rysunek~\ref{fig:spi1} pomija linię \emph{MISO}, gdyż~częstym zastosowaniem architektury jest~jednokierunkowe
przesyłanie rozkazów do~urządzeń podrzędnych.

\begin{figure}[H]
  \centering
  \copyrightbox{
    \includegraphics[width=0.7\linewidth]{media/analog_spi1}
  }{
    \textcopyright Analog Devices Inc.
  }
  \caption{Łącze bezpośrednie \emph{SPI}}
  \label{fig:spi1}
\end{figure}

Druga z~wykorzystywanych konfiguracji nie~wymaga wykorzystania więcej niż~jednej linii \emph{Chip Select},
w~zamian za~nieco bardziej skomplikowany schemat komunikacji wywołany niejednolitą ilością cykli zegara obsługującą
dane urządzenie podrzędne, w~tej~pracy otrzymuje miano metody ``cyklicznej''.
W~tym schemacie wejścia kolejnych urządzeń podrzędnych są~podłączone do~wyjść poprzedzających urządzeń.
Urządzenie nadzorcze wysyła serię komend dla~wszystkich z~urządzeń podrzędnych, wykorzystując cechę
składania~się przez~nie w~bufor kołowy, a~następnie aktywuje je~wszystkie jednocześnie zmieniając stan współdzielonej
linii \emph{Chip Select}.
Urządzenia podrzędne bez~zleconego zadania muszą otrzymać wartość zerową, która~jest~traktowana jako~brak danych.
Następnie urządzenie nadzorcze zbiera wszystkie odpowiedzi.
Ten~typ połączenia przedstawia rysunek~\ref{fig:spi2}.

\begin{figure}[H]
  \centering
  \copyrightbox{
    \includegraphics[width=0.7\linewidth]{media/analog_spi2}
  }{
    \textcopyright Analog Devices Inc.
  }
  \caption{Łącze cykliczne \emph{SPI}}
  \label{fig:spi2}
\end{figure}

\subsection{UART}\label{subsec:uart}

\emph{Universal Asynchronous Receiver-Transmitter}\thinspace\cite{uart} to~protokół komunikacyjny wykorzystujący
dwie~linie oraz~wspólne uziemienie.
Jest~to~protokół asynchroniczny, w~związku z~czym oba~uczestniczące w~transmisji urządzenia należy przygotować
do~komunikacji poprzez dobór parametrów komunikacyjnych na~te~same wartości, wspierane przez oba~urządzenia,
gdyż~protokół nie~wykorzystuje linii zegara synchronizującego.
Linia \emph{TX} jednego z~urządzeń jest~połączona do~linii \emph{RX} drugiego i~vice~versa.

Ustawienia, które muszą mieć na~urządzeniach zgodne wartości przedstawione zostały w~tabeli~\ref{tab:uartparameters}.

\begin{table}[H]
  \caption{Opis parametrów \emph{UART}}
  \label{tab:uartparameters}
  \begin{tabularx}{\linewidth}{@{}lX@{}}
    \textbf{Parametr}  & \textbf{Opis} \\
    \midrule
    Baud Rate          & Prędkość transmisji (ilość czasu przypisana jednemu bitowi). \\
    \midrule
    Data               & Ilość bitów w~wiadomości. \\
    \midrule
    Parity             & Tryb bitu parzystości (brak, parzysty, nieparzysty). \\
    \midrule
    Stop Bits          & Długość przerwy między wiadomościami. \\
    \midrule
    Flow Control       & Tryb współpracy z~wolniejszymi urządzeniami. \\
  \end{tabularx}
\end{table}

Wiadomość UART (rysunek~\ref{fig:uart}) składa~się z~zakończenia domyślnego trybu wysokiego nieczynnej linii
(stanem niskim) o~czasie trwania równym jednemu bitowi, a~następnie przesłanie \numrange{5}{9} bitów wiadomości
i~opcjonalnego bitu parzystości.
Następna wiadomość może zostać przesłana minimalnie po~czasie odpowiadającym \numrange{1}{2} bitom w~stanie spoczynku.

\begin{figure}[H]
  \centering
  \copyrightbox{
    \includegraphics[width=0.8\linewidth]{media/arduino_uart}
  }{
    \textcopyright Arduino
  }
  \caption{Schemat wiadomości \emph{UART}}
  \label{fig:uart}
\end{figure}

\section{Oprogramowanie}\label{sec:software}

\subsection{Raspberry Pi OS}\label{subsec:rpi-os}

\emph{Raspberry Pi OS}\thinspace\cite{rpios} (dawniej \emph{Raspbian}) to~dystrybucja \emph{GNU/Linux} bazująca
na~dystrybucji \emph{Debian}, utworzona i~przystosowana do~wykorzystania z~mikrokomputerami \emph{Raspberry Pi}.
W~momencie pisania pracy najnowsza wersja z~\num{1}~października~2025 opiera~się na~\emph{Debianie 13 Trixie},
ta~właśnie~wersja jest~wspierana przez skrypt instalacyjny i~środowisko laboratoryjne.

Dystrybucja ma~warianty przystosowane do~poszczególnych kategorii produktów \emph{Raspberry~Pi}, optymalizujące system
do~współpracy na~urządzeniach o~niskiej mocy obliczeniowej, ograniczonej wielkości, i~prędkości dysku (karty SD),
a~także zawierające pomocne skrypty konfiguracyjne do~ peryferiów płytek.
Jest~to oficjalny system operacyjny płytek \emph{Raspberry Pi}, w~związku z~czym można spodziewać~się możliwie
najlepszej stabilności i~wsparcia w~przypadku wystąpienia problemów.

\subsection{Bash}\label{subsec:bash}

\emph{Bash}\thinspace\cite{bash} to~jedna z~najpopularniejszych powłok (interpreterów poleceń) wykorzystywanych
w~systemach \emph{Unix}, \emph{Linux}, \emph{MacOS}, oraz \emph{Solaris}.
Poza obsługą poleceń użytkownika zawiera zaawansowane możliwości skryptowe, dzięki~czemu może~być wykorzystywana~do
szeroko rozumianej automatyzacji administracji systemem.

W~pracy wykorzystana właśnie w~tym celu, co~ułatwia osobie prowadzącej zajęcia wstępną konfigurację, a~także czyszczenie
środowiska między zajęciami -- załączony skrypt instalacyjny \ax{pi-labs.sh}{media/pi-labs.sh}[mimetype=text/plain]
wykorzystuje możliwości tego narzędzia do~pełnej konfiguracji płytki do~zajęć przy~użyciu jednej komendy.

\subsection{Python}\label{subsec:py}

\emph{Python}\thinspace\cite{python} to~interpretowany, wysokopoziomowy język programowania.
Jest znany z~przejrzystej składni, łatwości nauki i~rozbudowanego zestawu bibliotek, zarówno wbudowanych
jak~i~utworzonych przez~społeczność.
Dzięki~temu również ilość materiałów na~temat wykorzystania języka jest~bardzo~duża.

Ekspresywność języka pozwala na~wyrażenie szerokiego zakresu wyrażeń w~prosty sposób, a~także wykorzystanie wszystkich
popularnych paradygmatów programowania, co~jest dużą~zaletą w~trakcie nauki innych narzędzi -- dzięki tej~prostocie
nie~trzeba ``walczyć'' ze~składnią języka, a~można skupić~się na~przyswojeniu nowych informacji na~temat nowego
narzędzia.

Największą wadą języka jest~prędkość interpretera, która~nie~pozwala na~wykorzystania w~zastosowaniach
o~krytycznej wydajności.
Część osób jest~również przeciwna użyciu przez~język białych znaków do~definicji bloków semantycznych.

\subsection{Rpi-lgpio}\label{subsec:lgpio}

Wiązania \emph{rpi-lgpio}\thinspace\cite{lgpio} języka~C do~użytkowania interfejsu \emph{GPIO} zastępują bibliotekę
\emph{rpi-gpio}.
Wykorzystują te same nazwy importowane i~sygnatury w~celu zapewnienia kompatybilności wstecznej.
Istnieją z~powodu usunięcia interfejsu \emph{sysfs} do \emph{GPIO} z~nowszych wersji jądra systemowego.
Zamiast przestarzałego interfejsu sysfs wykorzystują nowszy plik urządzenia \emph{gpiochip}.
Podstawowe wykorzystanie prezentuje rysunek~\ref{fig:gpio}.

\begin{figure}[H]
  \lstinputlisting[language=python,label={lst:gpio}]{media/gpio.py}
  \caption{Podstawowe wykorzystanie \emph{GPIO} w języku \emph{Python}}
  \label{fig:gpio}
\end{figure}

\subsection{Marimo}\label{subsec:marimo}

\emph{Marimo}\thinspace\cite{marimo} jest~środowiskiem notatnikowym \emph{Python} mającym na~celu usunięcie niektórych
znaczących wad wcześniej istniejących rozwiązań\thinspace\cite{marimofaq} (tabela~\ref{tab:marimotraits}).
Poza oferowaniem równie zaawansowanej funkcjonalności notatnika architektura projektu gwarantuje cechy wpływające
na~wygodę i~stabilność użytkowania.

\begin{table}[H]
  \caption{Cechy notatnika \emph{marimo} wynikające z~braków w~innych rozwiązaniach}
  \label{tab:marimotraits}
  \begin{tabularx}{\linewidth}{@{}lX@{}}
    \textbf{Cecha}       & \textbf{Opis} \\
    \midrule
    Powtarzalność        & Kod widoczny w~notatniku \emph{Jupyter} niekoniecznie odpowiada obecnemu stanowi projektu,
                           co~sprawia,~że \qty{36}{\percent} notatników nie~jest powtarzalnych\thinspace\cite{jupyfail}.
                           Wykorzystanie notatnika \emph{marimo} rozwiązuje ten problem. \\
    \midrule
    Łatwość utrzymania   & Notatniki \emph{marimo} są~przechowywane jako~standardowe programy \emph{Python}, co~pozwala
                           m.in.~na ich~bezproblemowe wersjonowanie. \\
    \midrule
    Interaktywność       & Środowisko \emph{marimo} zawiera funkcjonalność tworzenia elementów interfejsu użytkownika,
                           automatycznie synchronizowanych w~kodzie przy~zmianie wartości. \\
    \midrule
    Wykorzystanie wtórne & Notatniki \emph{marimo} mogą~być wykorzystane z~poziomu linii komend jak~zwyczajne skrypty
                           w~języku \emph{Python}, można również zaimportować zdefiniowane w~nich symbole do~innych
                           plików. \\
    \midrule
    Udostępnianie        & Każdy notatnik \emph{marimo} może~również być~uruchomiony jako~interaktywna aplikacja
                           Internetowa wyświetlająca interfejs użytkownika bez~możliwości edycji kodu.
  \end{tabularx}
\end{table}

Środowisko \emph{marimo} wspiera zarówno tryb jasny, jak~i~ciemny interfejsu użytkownika; w~tej sekcji znajduje~się
przykład trybu ciemnego (rys.~\ref{fig:darkmarimo}), a~w~pozostałych przypadkach w~celu lepszego dopasowania
kolorystycznego rysunki będą~wykorzystywać tryb jasny.

\begin{figure}[H]
  \centering
  \includegraphics[width=0.8\linewidth]{media/marimo_dark}
  \caption{Tryb ciemny interfejsu graficznego notatnika \emph{marimo}}
  \label{fig:darkmarimo}
\end{figure}

\section{Wdrożenie rozwiązania}\label{sec:deployment}

\subsection{Udostępnianie}\label{subsec:share}

Pierwotną metodą udostępnienia elementu technicznego tej~pracy dyplomowej jest~skrypt \emph{Bash}, zawarty zarówno
w~tym pliku PDF (przeglądarki nie~obsługują poprawnie formatu, do~przejrzenia zagnieżdżonych plików należy~użyć programu
\emph{Adobe Acrobat Reader}), w~Archiwum Prac Dyplomowych, w~repozytorium archiwalnym\thinspace\cite{repo}, jak~i
na~mojej~stronie Internetowej\thinspace\cite{script}.
Zawiera~on wszystkie informacje niezbędne do~przygotowania i~uruchomienia środowiska laboratoryjnego na~płytce
\emph{Raspberry Pi} ze~świeżo zainstalowanym systemem opartym~na \emph{Debianie 13 Trixie}.
Dzięki użyciu stałych wersji programów ilość zmian niezbędnych do~przystosowania skryptu do~istniejących w~przyszłości
wersji systemu powinna~być minimalna.
Skrypt~ten może~być~również użyty po~zakończeniu laboratoriów do~wyczyszczenia środowiska od~zmian wprowadzonych
przez~studenta.

\subsection{Pliki dodatkowe}\label{subsec:addons}

Dla~zapewnienia wstępnej wiedzy na~temat współpracy ze~środowiskiem \emph{marimo}, poza~laboratoriami część praktyczna
pracy zawiera również plik wprowadzeniowy, którego fragment znajduje~się na~powyższym rysunku~\ref{fig:darkmarimo}.
Opisuje~on rozwiązania graficzne i~interakcje możliwe do~przeprowadzenia w~środowisku.

Do~pracy dołączony~jest również instruktażowy plik łączenia modułów wraz z~odpowiedziami na~zadania laboratoryjne
\ax{dla-prowadzacego.pdf}{out/dla-prowadzacego.pdf}[mimetype=application/pdf], pomagający osobie prowadzącej zajęcia
połączenie elementów sprzętowych laboratoriów oraz pomoc w~przypadku problemów z~rozwiązaniem zadania.

W~celu pokazania pełnej treści laboratoriów dołączony~jest również plik zawierający zdjęcia pełnej zawartości instrukcji
studenckich \ax{instrukcje.pdf}{instrukcje.pdf}[mimetype=application/pdf].

\subsection{Testy rozwiązań}\label{subsec:tests}

Zadania pośrednie przygotowane dla~studentów -- wymagające napisania kodu -- są~automatycznie sprawdzane
przy~wykorzystaniu możliwości środowiska notatnika \emph{marimo}.
Pozwala~to zaoszczędzić czas prowadzącego na~pomoc indywidualną; sprawdzenie końcowego działania programu gwarantuje
poprawne rozwiązanie wcześniejszych zadań (lub~indywidualną wiedzę studenta, jeśli~rozwiązanie jest~inne, ale~również
działa poprawnie).

Testy wykorzystują standardową bibliotekę języka, \emph{unittest}, w~celu największej przenoszalności.
Kod jest~napisany w~sposób niezawierający bezpośrednio prawidłowego rozwiązania, aby~utrudnić możliwe oszukiwanie
poprzez~sprawdzenie rozwiązania w~teście.
Całkowite zlikwidowanie takich sytuacji nie~jest możliwe, jednak również udowadnia wysoki poziom wiedzy studenta
o~języku \emph{Python}.

\subsection{Bezpieczeństwo}\label{subsec:security}

Przed pierwszym uruchomieniem skryptu z~nieznanego źródła należy zapoznać~się z~jego~treścią przy~użyciu edytora
tekstowego.
Nie~należy uruchamiać kodu, którego~treść jest~nam niezrozumiała.
Dotyczy~to załączonego do~pracy skryptu instalacyjnego.

Skrypt zawiera środki testujące prawidłowość wykonania, jednak w~każdym kodzie możliwe~są błędy i~niezaplanowane
wykorzystania.
Wykorzystane środki zapobiegawcze:
\begin{itemize}
  \item Wykluczenie sprzętu mogącego uszkodzić płytkę \emph{Raspberry Pi} wyższym napięciem;
  \item zawarte w~pracy, a~także w~nagłówku skryptu ostrzeżenie o~nieuruchamianiu nieznanego kodu;
  \item test otrzymania uprawnień administratorskich;
  \item użycie folderu tymczasowego przez~całą~długość działania kodu;
  \item sprawdzenie istnienia zmian wykonywanych przez~skrypt przed~ich~wykonaniem;
  \item wykorzystanie stałej wersji języka \emph{Python} i bibliotek;
  \item wykorzystanie środowiska wirtualnego;
  \item załączona~instrukcja połączeń dla~prowadzącego zajęcia;
  \item załączony~plik wprowadzający do~wykorzystania środowisk notatnikowych;
  \item automatyczne testy prawidłowości wykonania zadań przez~studenta;
  \item wyżej wymienione testy nie~uruchamiają kodu studenta.
\end{itemize}
