\frontmatter

% *************** Strona tytułowa ***************
\pagestyle{empty}
\pagenumbering{arabic}

\noindent

\AddToShipoutPictureBG*{
	\AtPageUpperLeft{\raisebox{-\height}{
		\includegraphics[height=0.23\paperwidth]{media/pbs_logo}
	}
}}

\vspace*{4cm}

\begin{tabular}{l}
   \HUGE PRACA DYPLOMOWA \\
   \HUGE INŻYNIERSKA \\
   \LARGE na~kierunku Informatyka Stosowana
\end{tabular}

\vfill

\begin{tabular}{p{15cm}}
    \Large\textbf{Opracowanie zestawu ćwiczeń laboratoryjnych wykorzystujących Raspberry~Pi} \\
	\\
	\large\textbf{Development of~a~set of~laboratory exercises using Raspberry~Pi}
\end{tabular}

\vfill

\begin{tabular}{l}
    \Large Remigiusz Dończyk \\
    \num{121179} \\
    \\
    \\
    dr.~inż.~Beata Marciniak \\
    \\
    Bydgoszcz, \today
\end{tabular}

\cleardoublepage

\pagestyle{plain}
\scriptsize

\section*{Metryka pracy dyplomowej}
\subsubsection*{Dane ogólne}
\begin{tabular}{p{3.5cm}p{11.5cm}}
    Nazwa Uczelni   & Politechnika Bydgoska im.~Jana i~Jędrzeja Śniadeckich \\
    Wydział         & Telekomunikacji, Informatyki i~Elektrotechniki \\
    Kierunek        & Informatyka Stosowana \\
    Tryb studiów    & niestacjonarne \\
    Dane autora     & Remigiusz Dończyk, \num{121179} \\
    Dane promotora  & dr.~inż.~Beata Marciniak
\end{tabular}

\subsubsection*{Dane dotyczące pracy dyplomowej}
\begin{tabular}{p{3.5cm}p{11.5cm}}
    Język pracy     & język polski [PL] \\
    Tytuł pracy     & Opracowanie zestawu ćwiczeń laboratoryjnych wykorzystujących \emph{Raspberry Pi}. \\
    Opis pracy      &%
        Celem pracy jest dobór zestawów laboratoryjnych i~opracowanie instrukcji do~maksymalnie \num{6}~ćwiczeń
        z~użyciem dostarczonych przez prowadzącego zestawów czujników oraz \emph{Raspberry~Pi}.
        Każde gotowe ćwiczenie powinno składać~się z~przykładowego programu w~języku \emph{Python}, pozwalającego
        na~konfigurację urządzeń, oraz~instrukcji przebiegu ćwiczenia.
        \begin{enumerate}
            \item Dobór zestawów czujników, różnych dla~każdego ćwiczenia.
            \item Przygotowanie programów w~języku \emph{Python} wykorzystujących zestawy czujników.
            \item Opracowanie instrukcji ćwiczeń do~każdego zestawu, zawierających przykładowe zadanie do~rozwiązania.
            \item Wykonanie rozwiązań zadań zdefiniowanych w~instrukcjach ćwiczeń.
            \item Przetestowanie działania rozwiązań i~bezpieczeństwa użytkowania.
        \end{enumerate} \\
    Typ pracy       & inżynierska \\
    Streszczenie    &%
        Praca dyplomowa jest rozwiązaniem problemu inżynierskiego na~temat wykorzystania mikrokomputera
        \emph{Raspberry Pi} wraz~z~zestawem modułów fizycznych do~opracowania serii zadań laboratoryjnych.
        Na~początku pracy zamieszczono informacje wspólne dla~laboratoriów: wymieniono dostępne moduły i~wybór
        spośród~nich zestawów laboratoryjnych; w~przeglądzie rynku wyróżniono istniejące alternatywy
        \emph{Raspberry~Pi}; opisano połączenia fizyczne i~logiczne wykorzystywane przez~użyte~moduły, a~także
        oprogramowanie składające~się na~wygodny w~użytkowaniu dla~prowadzącego i~studentów system, oraz~rozwiązania
        mające~wpływ na~bezpieczeństwo urządzeń.
        Dalsza część pracy jest~podzielona na~sekcje poświęcone każdemu z~opracowanych laboratoriów, prezentuje zestaw
        laboratoryjny wraz~z~rozwiązaniem zadania laboratoryjnego w~celu weryfikacji poprawności przygotowania. \\
    Słowa kluczowe  & komputer, \emph{Python}, \emph{Raspberry Pi}, czujniki, laboratoria
\end{tabular}

\newpage

\section*{Diploma thesis record}
\subsubsection*{General information}
\begin{tabular}{p{3.5cm}p{11.5cm}}
    University name           & Bydgoszcz University of~Science and~Technology \\
    Faculty                   & Telecommunications, Computer~Science and~Electrical~Engineering \\
    Field of study            & Applied Computer Science \\
    Mode of study             & part-time \\
    Author's information      & Remigiusz Dończyk, \num{121179} \\
    Supervisor's information  & dr.~inż.~Beata Marciniak
\end{tabular}

\subsubsection*{Data regarding the diploma thesis}
\begin{tabular}{p{3.5cm}p{11.5cm}}
    Language of thesis  & Polish [PL] \\
    Title of thesis     & Development of~a~set of~laboratory exercises using \emph{Raspberry Pi}. \\
    Description         &%
        The goal of~the~thesis is~the~selection of~laboratory exercise kits and~the~development of~up~to~\num{6}
        experiments using hardware sensors and~\emph{Raspberry Pi} boards provided by~the~instructor.
        Each laboratory should consist of~an~example \emph{Python} program, allowing for~the~configuration of~hardware,
        as~well as~an~exercise guide document.
        \begin{enumerate}
            \item Selection of~sensors, unique to~each exercise.
            \item Preparation of~\emph{Python} programs utilizing the~selected hardware.
            \item Development of~instructions for~each exercise, containing tasks to~be~completed.
            \item Execution of~solutions for~tasks contained in~the~instructions.
            \item Test of~the~solutions' functionality and~safety of~usage.
        \end{enumerate} \\
    Type of thesis      & Engineer’s \\
    Abstract            &%
      This~thesis is~the~solution to~an~engineering problem involving the~use~of a~\emph{Raspberry Pi} microcomputer
      and~a~set of~physical modules to~develop a~series of~laboratory exercises.
      At~the~beginning of~the~document is~information common to~the~laboratories: a~list of~the~available modules
      and~the~selection of~laboratory kits from~them; the~market survey, an~overview of~existing alternatives
      to~\emph{Raspberry Pi}; a~description of~the~physical and~logical connections used by~the~selected modules,
      as~well~as the~software making~up a~user-friendly system for~the~instructor and~students, along with~solutions
      affecting device security.
      The~latter part of~the~thesis is~divided into~sections dedicated to~each~laboratory, presents the~laboratory kit
      and~the~solution to~the~laboratory exercise to~verify its~correctness. \\
    Keywords            & computer, \emph{Python}, \emph{Raspberry Pi}, sensors, laboratories
\end{tabular}

\normalsize\clearpage
\tableofcontents
