\frontmatter

% *************** Strona tytułowa ***************
\pagestyle{empty}
\pagenumbering{arabic}

\noindent

\AddToShipoutPictureBG*{
	\AtPageUpperLeft{\raisebox{-\height}{
		\includegraphics[height=0.23\paperwidth]{media/pbs_logo}
	}
}}

\vspace*{4cm}

\begin{tabular}{l}
   \HUGE PRACA DYPLOMOWA\\
   \HUGE INŻYNIERSKA\\
   \LARGE na~kierunku Informatyka Stosowana\\
\end{tabular}

\vfill

\begin{tabular}{p{15cm}}
    \Large\bfseries
	\textbf{Opracowanie zestawu ćwiczeń laboratoryjnych wykorzystujących Raspberry~Pi}\\
	\\
	\large\bfseries
	\textbf{Development of~a~set of~laboratory exercises using Raspberry~Pi}\\
\end{tabular}

\vfill

\begin{tabular}{l}
    \Large Remigiusz Dończyk\\
    121179\\
    \\
    \\
    dr.~inż.~Beata Marciniak\\
    \\
    Bydgoszcz, \today\\
\end{tabular}

\cleardoublepage

\pagestyle{plain}
\scriptsize

\section*{Metryka pracy dyplomowej}
\subsubsection*{Dane ogólne}
\begin{tabular}{p{3cm}p{12cm}}
    Nazwa Uczelni   & Politechnika Bydgoska im.~Jana i~Jędrzeja Śniadeckich\\
    Wydział         & Telekomunikacji, Informatyki i~Elektrotechniki\\
    Kierunek        & Informatyka Stosowana\\
    Tryb studiów    & niestacjonarne\\
    Dane autora     & Remigiusz Dończyk, 121179\\
    Dane promotora  & dr.~inż.~Beata Marciniak\\
\end{tabular}

\subsubsection*{Dane dotyczące pracy dyplomowej}
\begin{tabular}{p{3cm}p{12cm}}
    Język pracy     & język polski [PL]\\
    Tytuł pracy     & Opracowanie zestawu ćwiczeń laboratoryjnych wykorzystujących Raspberry~Pi\\
    Opis pracy      &%
        Celem pracy jest dobór zestawów laboratoryjnych i~opracowanie instrukcji do~maksymalnie 6~ćwiczeń z~użyciem dostarczonych przez prowadzącego zestawów czujników oraz Raspberry~Pi.
        Każde gotowe ćwiczenie powinno składać~się z~przykładowego programu w~języku Python, pozwalającego na~konfigurację urządzeń, oraz~instrukcji przebiegu ćwiczenia.
        \begin{enumerate}
            \item Dobór zestawów czujników, różnych dla~każdego ćwiczenia.
            \item Przygotowanie programów w~języku Python wykorzystujących zestawy czujników.
            \item Opracowanie instrukcji ćwiczeń do~każdego zestawu, zawierających przykładowe zadanie do~rozwiązania.
            \item Wykonanie przykładowych rozwiązań zadań zdefiniowanych w~instrukcjach ćwiczeń.
            \item Przetestowanie działania rozwiązań i~bezpieczeństwa użytkowania.
        \end{enumerate}\\
    Typ pracy       & inżynierska\\
    Streszczenie    & TODO: Tekst streszczenia w języku polskim. Streszczenie pracy dyplomowej (max. pół strony) powinno zawierać omówienie zagadnień poruszanych w pracy. W części tej należy pokrótce scharakteryzować cel oraz podstawowe założenia pracy.\\
    Słowa kluczowe  & komputer, python, raspberry~pi, czujniki, laboratoria\\
\end{tabular}

\newpage

\section*{Diploma thesis record}
\subsubsection*{General information}
\begin{tabular}{p{3cm}p{12cm}}
    University name           & Bydgoszcz University of~Science and~Technology\\
    Faculty                   & Telecommunications, Computer~Science and~Electrical~Engineering\\
    Field of study            & Applied Computer Science\\
    Mode of study             & part-time\\
    Author's information      & Remigiusz Dończyk, 121179\\
    Supervisor's information  & dr.~inż.~Beata Marciniak\\
\end{tabular}

\subsubsection*{Data regarding the diploma thesis}
\begin{tabular}{p{3cm}p{12cm}}
    Language of thesis  & Polish [PL]\\
    Title of thesis     & Development of~a~set of~laboratory exercises using Raspberry~Pi\\
    Description         &%
        The goal of~the~thesis is~the~selection of~laboratory exercise kits and~the~development of~up~to~6 experiments using hardware sensors and~Raspberry~Pi boards provided by~the~instructor.
        Each laboratory should consist of~an~example Python program, allowing for~the~configuration of~hardware, as~well as~an~exercise guide document.
        \begin{enumerate}
            \item Selection of~sensors, unique to~each exercise.
            \item Preparation of~Python programs utilizing the~selected hardware.
            \item Development of~instructions for~each exercise, containing tasks to~be~completed.
            \item Execution of~example solutions for~tasks contained in~the~instructions.
            \item Test of~the~solutions' functionality and~safety of~usage.
        \end{enumerate}\\
    Type of thesis      & Engineer’s\\
    Abstract            & TODO: The abstract should be written in English and provide a concise summary of the thesis (maximum half a page). It should briefly describe the purpose and basic assumptions of the thesis.\\
    Keywords            & computer, python, raspberry~pi, sensors, laboratories\\
\end{tabular}

\normalsize\clearpage
\tableofcontents
