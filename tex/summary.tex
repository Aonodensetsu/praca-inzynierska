\chapter{Podsumowanie}\label{ch:summary}

Niniejsza praca inżynierska implementuje zestawy laboratoryjne wprowadzające studentów do~wykorzystania peryferiów
\emph{GPIO} z~komputerem jednopłytkowym \emph{Raspberry Pi}.
W~tym~celu dobrane moduły sprzętowe, a~także rozwiązania architektury oprogramowania są~dopasowane do~warunków uczelni,
pozwalając zarówno na~szybkie i~nieskomplikowane przygotowanie urządzeń do~zajęć -- co~ogranicza nakład powtarzalnych
czynności dla~osoby prowadzącej zajęcia -- jak~i~dokładnie udokumentowane i~spójne środowisko udostępnione studentom.
Laboratoria wykorzystują różne moduły fizyczne i~w~różny sposób wchodzą z~nimi w~interakcje z~poziomu kodu, co~pozwala
studentom na~nabycie unikalnych umiejętności w~każdym z~zestawów.

Poprawne rozwiązania zadań zostały przetestowane i~umieszczone w~załączniku dla~prowadzącego, co~w~razie wątpliwości
pozwoli na~uzupełnienie wszystkich przygotowanych zadań.
Skrypt instalacyjny zawiera~bezpośrednio kod notatnika, dzięki~czemu jakiekolwiek zmiany lub~uzupełnienia treści
laboratoriów mogą~być zintegrowane do~dalszych zajęć kopiując nową treść pomiędzy znaczniki graniczące zawartość pliku
-- rysunek~\ref{fig:labsh}.
W~ten~sam~sposób można rozszerzyć zakres pracy o~nowe laboratoria, zwiększając zakres dostępnego materiału.

\begin{figure}[H]
  \lstinputlisting[language=bash,firstline=2,label={lst:labsh}]{media/lab.sh}
  \caption{Populacja treści pliku}
  \label{fig:labsh}
\end{figure}

W~celu zapewnienia względów bezpieczeństwa środowisko laboratoryjne jest~dostępne jedynie z~lokalnej maszyny, jednak
oprogramowanie notatnika \emph{marimo} wspiera uruchamianie za~wstecznym proxy, co~może~pozwolić na~rozszerzenie
projektu o~prowadzenie zajęć zdalnych.

Obecnie sprawozdania pełnią funkcję archiwizacji pracy studentów, ale~ciekawą opcją do~rozważenia mogłoby~być
wykorzystanie uzupełnionych notatników laboratoryjnych w~tym~celu, gdyż wielkość tych plików jest~nieduża, a~zawierają
już informacje powielane przez~studentów do~edytora dokumentów \emph{Word} za~pomocą zrzutów ekranu.
